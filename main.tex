\documentclass{amsart}
\usepackage[utf8]{inputenc}
\usepackage{mymacros}

\title{Point scheme vs Hilbert scheme}
\date{\today}

\newtheorem{thm}{Theorem}
\newtheorem{cor}[thm]{Corollary}
\newtheorem{lem}[thm]{Lemma}
\newtheorem{prop}[thm]{Proposition}

\theoremstyle{definition}
\newtheorem{defn}[thm]{Definition}
\newtheorem{rem}[thm]{Remark}
\newtheorem{notn}[thm]{Notation}
\newtheorem{conj}[thm]{Conjecture}
\newtheorem{eg}[thm]{Example}

\begin{document}

\maketitle

We take $Q:= Q_{(6.1)}$ and $Q'$ to be $Q$ with the vertex $(0,0)$ removed along with arrows emanating from it; relations $I$ on $Q$ induce relations $I'$ on $Q'$.
We fix a generic enough ideal of relations $I$ in the dense torus once and for all.

The ideal $I$ gives us a collection of numbers $(\alpha_{ij}) \in \bG_m^{3 \times 3}$ so that the relations between the paths from $(i,0)$ and the vertices $(0,2)$ are:
\begin{equation}\label{eq:z}
    \begin{pmatrix}
    \alpha_{11}\, x_0 & \alpha_{12}\, y_0 & \alpha_{13}\, z_0 \\
    \alpha_{21}\, z_1 & \alpha_{22}\, x_1 & \alpha_{23}\, y_1 \\
    \alpha_{31}\, y_2 & \alpha_{32}\, z_2 & \alpha_{33}\, x_2
\end{pmatrix} \begin{pmatrix}
    x_0' \\ z_1' \\ y_2'
\end{pmatrix} =0.
\end{equation}
We use $M_\alpha$ to denote the matrix on the righthand side of Equation~(\ref{eq:z}).
Similarly we have $(\beta_{ij}), (\gamma_{ij}) \in \bG_m^{3 \times 3}$ and matrices $M_\beta, M_\gamma$ expressing relations between paths $(i,0)$ and the vertices $(1,2)$ and $(2,2)$, respectively.
To ease notation we rescale by an element of $\text{Aut}(\bk Q_{(6,1)})$ so that $\alpha_{ij}=1$ for all $i$ and $j$.

Before we address the relationship between point schemes and Hilbert schemes we strengthen the results discussed in the previous note on stability conditions.
In particular, we have the following result.

\begin{lem}
Given a generic ideal of relations $I$, the parameters $$\theta_1:= (-2,-2,-2,1,1,1,1,1,1) \quad \text{and} \quad \theta_2:= (-1,-1,-1,0,0,0,1,1,1)$$ give the same stability conditions on representations of dimension vector $\vec{1}$ on $Q$.
Furthermore, stability with respect to these conditions is equivalent to:
\begin{enumerate}
    \item[(2)] for every $i,j \in \{0,1,2\}$ there is a path $p$ from $(i,0)$ to $(j,2)$ for which $v_p\neq 0$.
\end{enumerate}
\end{lem}

\begin{proof}
Recall that stability with respect to $\theta_1$ is equivalent to:
\begin{enumerate}
    \item for every $k \in \{0,1,2\}$ there is an arrow $a$ with target $(k,1)$ for which $v_a \neq 0$. 
    \item for every $i,j \in \{0,1,2\}$ there is a path $p$ from $(i,0)$ to $(j,2)$ for which $v_p\neq 0$.
\end{enumerate}
We show that condition (2) implies (1).

Assume condition (2) holds.
This gives us the following:
\begin{enumerate}
    \item[(a)] no row is entirely zero,
    \item[(b)] no two columns can vanish simultaneously.
\end{enumerate}
Let the first two columns be the nonzero ones prescribed by condition (b).
To show condition (1) it remains to show that the third column is also nonzero.

Assume seeking a contradiction that the third column is zero.
This implies that the kernel of $M_\alpha$ contains a vectors of the form $(0,0,\lambda)^T$ for $\lambda \in \bC$.
However, this can not be the whole kernel of $M_\alpha$ since $(x_0',z_1',y_2')^T$ belongs to it; having the form $(0,0,\lambda)^T$ would then contradict condition (2).
Therefore, the kernel also contains a vector $(\mu, \nu, 0)^T$ with $\mu, \nu \in \bG_m$. 

Given conditions (a) and (b), without loss of generality we may assume $x_0 \neq 0$ and $x_1 \neq 0$.
Since our relations ideal $I$ is generic, we may assume that our $(\beta_{ij})$ satisfy $\beta_{11}\beta_{22}- \beta_{12}\beta_{21} \neq 0$.
Note that this equation on the $(\beta_{ij})$s is homogenous when one scales $\alpha_{ij}$ to 1.
This assumption about $M_\beta$ implies that the only vectors in the kernel of $M_\beta$ are of the form $(0,0,\lambda)^T$ for $\lambda \in \bC$ but again $(x_0',z_1',y_2')^T$ belongs to it and this contradicts condition (2).

A similar argument applied to matrices expressing relations between paths from $(0,0)$ and $(1,0)$ to the vertices $(i,2)$ gives us that condition (2) also implies: for every $k \in \{0,1,2\}$ there is an arrow $a$ with source $(k,1)$ for which $v_a \neq 0$.
Along with the proof in the stability note this completes the proof.
\end{proof}

Fix $\theta= \theta_2$ and define $\theta':= (-1,-1,0,0,0,1,1,1)$. 
We compare the moduli space of $\theta$-stable representations on our quiver $(Q,I)$ with a closed subvariety of moduli space of $\theta'$-stable representations on $(Q',I')$.
We will denote the moduli spaces $\cN$ and $\cN'$ respectively.

First we specify the closed locus in question.
The relations between the paths from $(0,0)$ and the vertices $(i,2)$ are:
\begin{equation}\label{eq:z}
    \begin{pmatrix}
    \alpha_{11}\, x_0' & \alpha_{12}\, z_1' & \alpha_{13}\, y_2' \\
    \beta_{11}\, y_0' & \beta_{12}\, x_1' & \beta_{13}\, z_2' \\
    \gamma_{11}\, z_0' & \gamma_{12}\, y_1' & \gamma_{13}\, x_2'
\end{pmatrix} \begin{pmatrix}
    x_0 \\ y_0 \\ z_0
\end{pmatrix} =0.
\end{equation}
We call the first factor of the product of matrices above $M_1$.
Similarly we get a matrix $M_2$ from relations between the paths from $(1,0)$ and the vertices $(i,2)$.
The closed subvariety of $\cN'$ will then given by $Z:= \{\det M_1=0\} \subset \cN'$.


\begin{lem}\label{lm:rank}
The $\rank(M_\alpha) = 2$ for any point $\bfp' \in \cN'$.
\end{lem}

\begin{proof}
Similar to the proof of Lemma 1 above.
%We may assume that $I$ lies in the dense torus of the moduli of relations.
%We may then rescale by an element of $\text{Aut}(\bk Q_{(6,1)})$ so that $\alpha_{ij}=1$ for all $i$ and $j$.
%
%Our choice of stability parameter gives us the following:
%\begin{enumerate}
%    %\item $(x_0', y_0', z_0')^T\neq 0$, this follows since  $\theta'_{(0,1)}<0$,
%    \item two columns can not vanish simultaneously,
%    \item no row of $M$ is entirely zero since $\theta'_{(i,2)}>0$ for all $i \in \{0,1,2\}$.
%\end{enumerate}
%These observations, in particular, rule out that $M$ is the zero matrix.
%
%Now assume $\rank(M)=1$.
%Our assumption gives us that the columns of $M$ are scalar multiples of each other.
%Combining our two observations above we get that at least one other column, say $(z_1', x_1', y_1')^T$ is non-zero.
%Since $(\beta_{ij}) \in \bG_m^{3 \times 3}$ is generic, the only vectors in the kernel of $M_\beta$ are of the form $(0,0,\lambda)$ for $\lambda \in \bC$.
%However, we have that $(z_1, x_1, y_1)^T$ is in the kernel of $M_\beta$ but $\theta'$ stability (in particular, the condition that at least two paths are nonzero from $(1,0)$ to $(0,2)$) implies it can not be of the prescribed form.
%We get a contradiction.
\end{proof}

We have a natural morphism $f \colon \cN \rightarrow \cN'$ forgetting the linear maps whose source is the vector space at $(0,0)$.
We first observe that the image of $f$ lies in $Z$.
The vector $(x_0, y_0, z_0)^T$ is non-zero by our choice of stability parameter; it gives us a non-zero vector in the kernel of $M_1$ and $\det M_1 = 0$.

Lemma~\ref{lm:rank} gives us $f^{-1}$ as follows. 
For every $p' \in \cN'$ we have $\rank(M_1)=2$.
Therefore is a unique nonzero vector in $\ker(M_1)$ up to scalar.
Taking this vector to be $(x_0, y_0, z_0)^T$ gives us a point in $\bfp$.
The scalar is taken care of the by the extra $\bG_m$ in the gauge group used to define $\cN$.

These on the face of it are rational maps.
The fact that $\theta$-stable points go to $\theta'$-stable points since condition (2) remains true when we forget the vertex $(0,0)$.
For the other direction, we observe that two of the components of $(x_0, y_0, z_0)^T$ can not vanish simultaneously since that would imply one of the rows of $M_1$ is zero but that is not allowed by the proof of Lemma 1.
\end{document}
