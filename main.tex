\documentclass{amsart}
\usepackage[utf8]{inputenc}
\usepackage{mymacros}

\title{Point scheme vs Hilbert scheme}
\date{\today}

\newtheorem{thm}{Theorem}
\newtheorem{cor}[thm]{Corollary}
\newtheorem{lem}[thm]{Lemma}
\newtheorem{prop}[thm]{Proposition}

\theoremstyle{definition}
\newtheorem{defn}[thm]{Definition}
\newtheorem{rem}[thm]{Remark}
\newtheorem{notn}[thm]{Notation}
\newtheorem{conj}[thm]{Conjecture}
\newtheorem{eg}[thm]{Example}

\begin{document}

\maketitle

We take $Q:= Q_{(6.1)}$ and $Q'$ to be $Q$ with the vertex $(0,0)$ removed along with arrows emanating from it; relations $I$ on $Q$ induce relations $I'$ on $Q'$.

Before we address the relationship between point schemes and Hilbert schemes we strengthen the results discussed in the previous note on stability conditions.
In particular, we have the following result.

\begin{lem}
Given a generic ideal of relations $I$, the parameters $$\theta_1:= (-2,-2,-2,1,1,1,1,1,1) \quad \text{and} \quad \theta_2:= (-1,-1,-1,0,0,0,1,1,1)$$ give the same stability conditions on representations of dimension vector $\vec{1}$ on $Q$.
\end{lem}

\begin{proof}
Recall that stability with respect to $\theta_1$ is equivalent to:
\begin{enumerate}
    \item for every $k \in \{0,1,2\}$ there is an arrow $a$ with target $(k,1)$ for which $v_a \neq 0$. 
    \item for every $i,j \in \{0,1,2\}$ there is a path $p$ from $(i,0)$ to $(j,2)$ for which $v_p\neq 0$.
\end{enumerate}
We show that condition (2) implies (1).

Assume condition (2) holds.
Fix a generic ideal of relations $I$.
This gives us a collection of numbers $(\alpha_{ij})$ so that the relations between the paths from $(i,0)$ and the vertices $(0,2)$ are:
\begin{equation}\label{eq:z}
    \begin{pmatrix}
    \alpha_{11}\, x_0 & \alpha_{12}\, y_0 & \alpha_{13}\, z_0 \\
    \alpha_{21}\, z_1 & \alpha_{22}\, x_1 & \alpha_{23}\, y_1 \\
    \alpha_{31}\, y_2 & \alpha_{32}\, z_2 & \alpha_{33}\, x_2
\end{pmatrix} \begin{pmatrix}
    x_0' \\ z_1' \\ y_2'
\end{pmatrix} =0.
\end{equation}
We may assume that $I$ lies in the dense torus of the moduli of relations.
We may then rescale by an element of $\text{Aut}(\bk Q_{(6,1)})$ so that $\alpha_{ij}=1$ for all $i$ and $j$.

Condition (2) then gives us the following:
\begin{enumerate}
    \item[(a)] no row is entirely zero,
    \item[(b)] no two columns can vanish simultaneously.
\end{enumerate}
With out loss of generality, conditions (a) and (b) give us that $x_0 \neq 0$ and $x_1 \neq $
Since $(\beta_{ij}) \in \bG_m^{3 \times 3}$ is generic, the only vectors in the kernel of $M_\beta$ are of the form $(0,0,\lambda)$ for $\lambda \in \bC$
\end{proof}

Here we compare the moduli space of $\theta$-stable representations on our quiver $(Q,I)$ with a closed subvariety of moduli space of $\theta'$-stable representations on $(Q',I')$.
We will denote the moduli spaces $\cN$ and $\cN'$ respectively.
We give an isomorphism between them when $I$ is generic enough.

First we specify the closed locus in question.
Fix a generic ideal of relations $I$.
This gives us a collection of numbers $(\alpha_{ij})$ so that the relations between the paths from $(0,0)$ and the vertices $(i,2)$ are:
\begin{equation}\label{eq:z}
    \begin{pmatrix}
    \alpha_{11}\, x_0' & \alpha_{12}\, z_1' & \alpha_{13}\, y_2' \\
    \alpha_{21}\, y_0' & \alpha_{22}\, x_1' & \alpha_{23}\, z_2' \\
    \alpha_{31}\, z_0' & \alpha_{32}\, y_1' & \alpha_{33}\, x_2'
\end{pmatrix} \begin{pmatrix}
    x_0 \\ y_0 \\ z_0
\end{pmatrix} =0.
\end{equation}
We call the first factor of the product of matrices above $M_\alpha$.
Similarly we get a matrix $M_\beta$ from relations between the paths from $(1,0)$ and the vertices $(i,2)$.
The closed subvariety of $\cN'$ will then given by $Z:= \{\det M_\alpha=0\} \subset \cN'$.


\begin{lem}\label{lm:rank}
If our ideal of relations $I$ is generic enough then $\rank(M_\alpha) = 2$ for any point $\bfp' \in \cN'$.
\end{lem}

\begin{proof}
We may assume that $I$ lies in the dense torus of the moduli of relations.
We may then rescale by an element of $\text{Aut}(\bk Q_{(6,1)})$ so that $\alpha_{ij}=1$ for all $i$ and $j$.

Our choice of stability parameter gives us the following:
\begin{enumerate}
    %\item $(x_0', y_0', z_0')^T\neq 0$, this follows since  $\theta'_{(0,1)}<0$,
    \item two columns can not vanish simultaneously,
    \item no row of $M$ is entirely zero since $\theta'_{(i,2)}>0$ for all $i \in \{0,1,2\}$.
\end{enumerate}
These observations, in particular, rule out that $M$ is the zero matrix.

Now assume $\rank(M)=1$.
Our assumption gives us that the columns of $M$ are scalar multiples of each other.
Combining our two observations above we get that at least one other column, say $(z_1', x_1', y_1')^T$ is non-zero.
Since $(\beta_{ij}) \in \bG_m^{3 \times 3}$ is generic, the only vectors in the kernel of $M_\beta$ are of the form $(0,0,\lambda)$ for $\lambda \in \bC$.
However, we have that $(z_1, x_1, y_1)^T$ is in the kernel of $M_\beta$ but $\theta'$ stability (in particular, the condition that at least two paths are nonzero from $(1,0)$ to $(0,2)$) implies it can not be of the prescribed form.
We get a contradiction.
\end{proof}

We have a natural morphism $f \colon \cN \rightarrow \cN'$ forgetting the linear maps whose source is the vector space at $(0,0)$.
We first observe that the image of $f$ lies in $Z$.
The vector $(x_0, y_0, z_0)^T$ is non-zero by our choice of stability parameter; it gives us a non-zero vector in the kernel of $M_\alpha$ and $\det M_\alpha = 0$.

Lemma~\ref{lm:rank} gives us $f^{-1}$ as follows. 
For every $p' \in \cN'$ we have $\rank(M_\alpha)=2$.
Therefore is a unique nonzero vector in $\ker(M_\alpha)$ up to scalar.
Taking this vector to be $(x_0, y_0, z_0)^T$ gives us a point in $\bfp$.
The scaling is taken care of the by the extra $\bG_m$ in the gauge group used to define $\cN$.

These on the face of it are rational maps.
The fact that $\theta$-stable points go to $\theta'$-stable points and vice versa may be observed if we view $\cN'$ as the contraction of $\cN$ along the edge $x_0$.
\end{document}
